%% -----------------------------------------------------------------------------
%% 
%% 
%% RUnit S4 class layout 
%%
%% $Id$
%%
%% Copyright (c) 2006  Epigenomics AG Berlin - All rights reserved
%% THIS IS PROPRIETARY SOURCE CODE of Epigenomics AG Berlin
%%
%% -----------------------------------------------------------------------------
%%
\documentclass[a4paper,10pt]{article}

\usepackage{epsfig}

%opening
\title{RUnit S4 Class Layout}
\author{Matthias Burger, and Thomas K\"onig}

\begin{document}

\maketitle

\begin{abstract}
Proposal for S4 classes for the RUnit package.
The intended scope of new/refactored functionality to be covered by the re-design is as
follows:
\begin{itemize}
\item internal representation of TestCase/TestFileTestSuite/TestSuites 
\item internal representation of
      TestCase/TestFile/TestSuite/TestSuites result data
\end{itemize}
\end{abstract}

\tableofcontents

\section{Design Goals}

\begin{itemize}

\item interface
\begin{itemize}
\item all public functions should stay, have the same arguments and
      behave as previously defined. That is the UI for defineTestSuite,
      runTestSuite, printTextProtocol, printHtmlProtocol should not be
      changed.
\item run time performance should be closely monitored and should not
      decrease by more than 15\%.
\end{itemize}

\item class hierarchy
\begin{itemize}
\item simple in the implementation
\item flexible in usage
\item use Array concept
\item allow/implement clusterApply distribution on nodes/recombination
      of results to one result object
\item compatible to S-PLUS 7 (if feasibile)
\end{itemize}

\end{itemize}


\section{Class Description}

\subsection{TestSuite}
This class records the relevant information to setup and execute a
test case runner. It stores the directory paths to consider, the
regular expression defining the test case files and test case
functions to execute, and the RNG methods to use as default.

\subsection{TestLogger}
Container class used by the test case runner (runTestSuite or
idirectly runTestFile) to record all intermediate test execution
details. In particular it contains the slot TestResultData as the 'top
level' class object recording all test case execution results.

\subsection{TestResultData}
'Top level' class object storing the test case execution results as a
hierarchy: TestSuits -> TestSuite -> SourceFile -> TestCase.

\subsection{TestSuiteTestResultData}
Container class for test case execution results for one complete test
suite (usually one package or one type (ie platform) of integration
tests). {\tt TestSuiteTestResultDataArray} derived from abstract {\tt
Array} class to store several test suite result objects.


\subsection{SourceFileTestResultData}
Container class for test case execution results for one test
case file (usually tests for one class or one project - for a given platform -
integration test)
tests). {\tt SourceFileTestResultDataArray} derived from abstract {\tt
Array} class to store several test file result objects. 


\subsection{TestCaseTestResultData}
Container class for test case execution results for one test
case. {\tt TestCaseTestResultDataArray} derived from abstract {\tt
Array} class to store several test case result objects. 


\subsection{Array}
Abstract base class for all containers of identical class objects,
implemented for convenience to provide all methods and the template
constructor for derived classes in one class.

\subsection{InspectTracker}
To be discussed: should we also refactor and rewrite the inspect and
tracker mechanism to make use of S4 classes and hiearchies?
\clearpage

\section{Class Diagram}
\begin{center}
\begin{figure}[htb]
\epsfig{file=RUnit_S4, width=\textwidth}
%\includegraphics[RUnit_S4]{width=6, height=7}
\caption{RUnit S4 class diagram (test case setup and execution only).}
\end{figure}
\end{center}

\end{document}
